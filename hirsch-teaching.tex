\documentclass{article}

\usepackage[utf8]{inputenc}
\usepackage[T1]{fontenc}
\usepackage[a4paper, top=0cm, margin=2cm, bottom=2cm]{geometry}
\usepackage[numbers,sort&compress,square]{natbib}
\usepackage{amsmath}
\usepackage{amssymb}
\usepackage{amsthm}
\usepackage{bbm}
\usepackage{bbold}
\usepackage{stmaryrd}
\usepackage{mathtools}
\usepackage{mathpartir}
\usepackage[dvipsnames]{xcolor}
\usepackage{pl-syntax}
\usepackage{xspace}
\usepackage{suffix}
\usepackage{turnstile}
\usepackage{multicol}

\usepackage[hidelinks]{hyperref}


% Notes
\newcommand{\uncertain}[1]{{\color{red} #1}\xspace}
\newcommand{\newcommenter}[3]{%
  \newcommand{#1}[1]{%
    \textcolor{#2}{\small\textsf{[{#3}: {##1}]}}%
  }%
}
\definecolor{darkgreen}{rgb}{0,0.7,0}
\newcommenter{\akh}{purple}{AKH}

% AMSTHM Setup
\newtheorem{thm}{Theorem}
\newtheorem{lem}{Lemma}
\newtheorem{cor}{Corollary}
\newtheorem{conj}{Conjecture}
\newtheorem{inv}{Invariant}
\theoremstyle{definition}
\newtheorem{defn}{Definition}

\author{Andrew K. Hirsch}
\title{Teaching Statement}
\date{}

\bibliographystyle{plainnat}

\begin{document}
\maketitle

Two anecdotes deeply influenced my teaching philosophy.
In one, I was a student, in the other I was a teacher.
Together, these anecdotes taught me two principles of teaching: \textbf{creativity} and \textbf{purpose}.
Using these principles has made me a much better teacher than I would otherwise be.

The first anecdote takes place when I started college.
All through high school I had hated math class, so I was nervous as a new computer-science major that I would not enjoy the math requirements.
In order to test my aptitude for my new major, I decided to take the math department's introductory linear-algebra course during my first semester.
It turned out to be a proof-based course about linear spaces, rather than a course on real-valued matrices.
This immediately clicked with me.
Proofs required \textbf{creativity}, both for the proof strategy and for the writing.
Moreover, they also seemed to me to have more \textbf{purpose} than problems in my previous courses: wanting to know \emph{why} something was true made more sense to me than crunching numbers.
Of course, these impressions were helped by the fun-loving and compassionate professor.

The second anecdote starts years later, when I was in graduate school.
I volunteered to teach programming after hours in a local elementary school along with several other Ph.D. students in my department.
We based the course material on \href{https://www.code.org}{code.org}, and so programmed a ``robot'' with a block-based visual language.
However, we decided that instead of doing exercises on the computer, we would instead have the kids program a human ``robot:'' first a teacher, then their friends in small groups.
This immediately engaged the kids, largely because it connected with their \textbf{purpose}: they wanted to play with their friends more than play a boring maze game on the computer.
Sadly, the school eventually pressured us to get the kids on computers; immediately, they found games they would rather play than the one we wanted them to.

These principles influenced my presence in the collegiate classroom as well as the elementary one.
When I was a teaching assistant for Cornell's undergraduate programming-languages class, I taught a section on proof writing.
In that section, I emphasized the purpose of proofs (to convince a reader) and the creativity required to write good proofs, emphasizing further the need to know your audience.
That emphasis made a notable difference in their proof-writing skills.

The principles have also informed my curriculum design.
When I was head teaching assistant for Cornell's functional-programming course, I helped revamp the material on formal methods to use the Coq proof assistant.
While this required a large amount of scaffolding work due to the advanced nature of Coq, students felt it was much more engaging than the traditional material.
First, they would generally rather program than prove pencil-and-paper theorems.
Second, the ability of Coq to prove the correctness of real programs helped them understand the purpose of formal methods.

Even when I did not get face-to-face communication with most students, I applied the principles of creativity and purpose.
When I was grading for Cornell's course \emph{Category Theory for Computer Scientists}, my job was to give feedback on weekly homeworks which emphasized not only proving theorems about category theory, but also applying category theory to problems in computing.
I made sure to give feedback not only on the correctness of solutions, but also on writing.
This feedback was received well by both students and faculty: I earned a TA award for my work on that course.

Going forward, I plan to continue to abide by these principles when bringing my research to the classroom.
Since my research emphasizes using abstract ideas from programming languages and logic to study decentralized programming, there are a number of course I could teach immediately.
For undergraduates, I could teach courses on functional programming, programming languages, and logic.
For graduate students, I can again teach programming languages, logic, and semantics; I can also teach specialist courses on category theory and language-based security.
With more preparation and effort, I could teach general security courses along with courses in verification and concurrency theory.
For each of these courses, I plan to use connections with the real world to give purpose to our studies, and to emphasize creative solutions in assignments.

\section*{Mentoring and Advising}

Much of a professor's teaching takes place outside of the classroom through mentoring and advising.
I have been lucky enough to mentor several younger students starting in graduate school.
My last year at Cornell I worked with a student who was just starting his graduate career.
I not only helped him find his bearings in graduate school, but also wrote a paper with him.
Since I have come to MPI, I have collaborated with two students here and one student at the University of Pennsylvania, again serving as a mentor.
Moreover, I have also served as a mentor to two graduate students and an undergraduate through SIGPLAN-M.
My advising philosophy advocates two principals: \textbf{purpose} and \textbf{openness}.

Purpose refers to working with students on both their career goals and their research goals.
I have regular conversations with my mentees about their career goals.
For instance, if a student wants to teach or go into industry, we should work on providing different opportunities than if they want to become research faculty.
Research goals likewise reach outside of one project: my goal is to help students build a research \emph{agenda}.
Thus, I have worked with students to not only decide on one project, but also to understand how it could fit into possible broader research agendas.

Openness refers to bidirectional communication with mentees.
This communication builds the sort of scaffolding that forms the foundation of the advising relationship.
While students always need independence, at the beginning an advisor has an active role in picking research projects, making arguments, and choosing between opportunities.
Over time, however, students need less and less active advising.
Thus, while I always aim to form a cohesive team with my students, over time the team relationship changes as the student takes more of a leadership role.
For me, watching a student take ownership of their research and their career is one of the most-fulfilling experiences possible.
\end{document}

%%% Local Variables:
%%% mode: latex
%%% TeX-master: t
%%% eval: (setenv "TEXINPUTS" ".::$TEXMF/tex/::./latex-pl-syntax/")
%%% End:
