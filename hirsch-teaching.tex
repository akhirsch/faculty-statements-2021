\documentclass{article}

\usepackage[utf8]{inputenc}
\usepackage[T1]{fontenc}
\usepackage[a4paper, top=0cm, margin=2cm, bottom=2cm]{geometry}
\usepackage[numbers,sort&compress,square]{natbib}
\usepackage{amsmath}
\usepackage{amssymb}
\usepackage{amsthm}
\usepackage{bbm}
\usepackage{bbold}
\usepackage{stmaryrd}
\usepackage{mathtools}
\usepackage{mathpartir}
\usepackage[dvipsnames]{xcolor}
\usepackage{pl-syntax}
\usepackage{xspace}
\usepackage{suffix}
\usepackage{turnstile}
\usepackage{multicol}

\usepackage[hidelinks]{hyperref}


% Notes
\newcommand{\uncertain}[1]{{\color{red} #1}\xspace}
\newcommand{\newcommenter}[3]{%
  \newcommand{#1}[1]{%
    \textcolor{#2}{\small\textsf{[{#3}: {##1}]}}%
  }%
}
\definecolor{darkgreen}{rgb}{0,0.7,0}
\newcommenter{\akh}{purple}{AKH}

% AMSTHM Setup
\newtheorem{thm}{Theorem}
\newtheorem{lem}{Lemma}
\newtheorem{cor}{Corollary}
\newtheorem{conj}{Conjecture}
\newtheorem{inv}{Invariant}
\theoremstyle{definition}
\newtheorem{defn}{Definition}

\author{Andrew K. Hirsch}
\title{Teaching Statement}
\date{}

\bibliographystyle{plainnat}

\begin{document}
\maketitle

Two anecdotes helped define my teaching philosophy: one took place while I was a student, the other when I was a teacher.
Together, these taught me three principles of teaching: \textbf{creativity, purpose, and engagement}.
In my teaching, I have focused on operationalizing these principles, making me a much better teacher than I would otherwise be.

The first anecdote starts when I started college.
All through high school I had hated math class, so I was nervous as a new computer-science major that I would not be able to handle the math requirements.
In order to test this out, I decided to take linear algebra via the math department in my first semester.
It turned out to be a proof-based course about linear spaces, rather than a course on real-valued matrices.
This immediately clicked with me.
Proof required \textbf{creativity}, both for the proof strategy and as a writer.
Moreover, they also seemed to me to have more \textbf{purpose} than problems in my previous courses: wanting to know \emph{why} something was true made more sense to me than crunching numbers.
Of course, these impressions were made with the help of a fun-loving, compassionate, and \textbf{engaging} professor.

The second anecdote takes place years later, when I was in graduate school.
I volunteered to teach programming after hours in a local elementary school along with several other Ph.D. students in my department.
While we wanted to base the course material after \href{https://www.code.org}{code.org}: blocks containing basic commands controlling a ``robot'' character.
However, we decided that instead of doing exercises on the computer, we would instead have the kids program a human ``robot:'' first a teacher, then their friends in small groups.
This immediately \textbf{engaged} the kids, largely because it connected with their \textbf{purpose}: they wanted to play with their friends more than play a boring maze game on the computer.
Sadly, we were eventually pressured by the schools to get the kids on computers; immediately, they found games they would much rather play than the one we wanted them to.

These principles have worked influenced my presence in the collegiate classroom as well as the elementary one.
When teaching assistant for Cornell's undergraduate programming-languages class, I taught a section on proof writing.
In that section, I emphasized the purpose of proofs (to convince a reader) and the creativity required to write good proofs, emphasizing further the need to know your audience.
That emphasis made a notable difference in their proof-writing skills.

The principles have also influenced my curriculum design.
When I was head teaching assistant for Cornell's functional-programming course, I helped revamp the material on formal methods to use the proof assistant Coq.
While this required a large amount of scaffolding work due to the advanced nature of Coq, students felt it was much more engaging than the previous material.
First, they would generally rather program than prove pencil-and-paper theorems.
Second, the ability of Coq to prove real programs correct helped them understand the purpose of formal methods better than the traditional material had.

Even when I did not get face-to-face communication with most students, I applied the three principles of creativity, purpose, and engagement.
When I was grading for Cornell's course \emph{Category Theory for Computer Scientists}, my job was to give feedback on weekly homeworks which emphasized not only proving theorems about category theory, but also applying category theory to problems in computing.
I made sure to give feedback not only on the correctness of solutions, but also on writing and connections with computing.
This feedback was received well by both students and faculty: I ended up earning a TA award from Cornell for that course.

Going forward, I plan to continue to abide by these principles when bringing my research to the classroom.
Since my research emphasizes using abstract ideas from programming-languages and logic to study real-world problems in decentralized programming, there are a number of course I could teach immediately.
For undergraduates, I could teach courses on functional programming, programming languages, and logic.
For graduate students, I can again teach programming languages, logic, and semantics; I can also teach specialist courses on category theory and language-based security.
With more preparation and effort, I could teach general security courses along with courses in verification and concurrency theory.
For each of these courses, I plan to use connections with the real world to give what we are studying purpose, and to emphasize creative solutions in assignments.

\section*{Mentoring and Advising}

A large amount of the teaching a professor does takes place outside of the classroom, through mentoring and advising.
I have been lucky enough to work as a mentor in several capacities, starting in graduate school.
My last year at Cornell I worked with a student who was just starting his graduate career, writing a paper with him and helping him find his bearings in graduate school.
Since I have come to the Max~Planck Institute, I have collaborated with two students here and one student at the University of Pennsylvania, also serving as a mentor when needed.
Moreover, I have also served as a mentor through SIGPLAN-M, offering mentoring and help to two graduate students and an undergraduate.
My advising philosophy can be summed up through two principals: \textbf{purpose} and \textbf{openness}.

Purpose refers to working with students on their career goals and, more specifically, on their research goals.
I try to have conversations regularly with the students I am mentoring on their career goals.
For instance, if a student wants to teach or go into industry, we should work on getting them different opportunities than if they want to become research faculty.
Research goals likewise reach outside of one project: my goal is to help students build a research \emph{agenda}.
Thus, I have worked with students to not only build one project, but also to understand how it could fit into possible broader research agendas.
For example, one student and I recently finished writing a paper together; since there were multiple dangling threads, we spent a long meeting discussing how different interests might lead to different projects.

Openness refers to communication with students and mentees in both directions.
I try to set expectations with my mentees, both in terms of my time and in terms of how much effort they should be putting into a project.
This communication is especially important around paper writing.
For instance, I was recently working with two students on pursuing close deadlines.
One project was being submitted for the first time, while the other was being resubmitted.
I had to let both know how much time I had for their project given how much I had to focus on the other; neither could get my full attention.

Moreover, helping students stretch their writing capabilities while providing the support and encouragement they need requires constant communication.
For instance, I work with students on putting together a story for a paper and outlining an argument for a section even in their first paper.
However, students naturally take time to understand the audiences of different venues, and more generally to build the empathy required to help a reader build intuition for a piece of work.
I thus work with students to scaffold their writing capabilities, filling in the pieces they can't yet and always helping them move on to the next goal.

This sort of scaffolding forms the foundation of the advising relationship.
While students always need independence, at the beginning an advisor has an active role in picking research projects, making arguments, and choosing between opportunities.
Over time, however, students need less and less active advising.
Thus, while I always aim to form a cohesive team with my students, over time the team relationship changes as the student takes over more and more leadership roles.
Watching a student take leadership of their research and their career is one of my most-fulfilling experiences.
\end{document}

%%% Local Variables:
%%% mode: latex
%%% TeX-master: t
%%% eval: (setenv "TEXINPUTS" ".::$TEXMF/tex/::./latex-pl-syntax/")
%%% End:
