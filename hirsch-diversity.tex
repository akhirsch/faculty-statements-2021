\documentclass{article}

\usepackage[utf8]{inputenc}
\usepackage[T1]{fontenc}
\usepackage[a4paper, top=0cm, margin=2cm, bottom=2cm]{geometry}
\usepackage[numbers,sort&compress,square]{natbib}
\usepackage{amsmath}
\usepackage{amssymb}
\usepackage{amsthm}
\usepackage{bbm}
\usepackage{bbold}
\usepackage{stmaryrd}
\usepackage{mathtools}
\usepackage{mathpartir}
\usepackage[dvipsnames]{xcolor}
\usepackage{pl-syntax}
\usepackage{xspace}
\usepackage{suffix}
\usepackage{turnstile}
\usepackage{multicol}

\usepackage[hidelinks]{hyperref}


% Notes
\newcommand{\uncertain}[1]{{\color{red} #1}\xspace}
\newcommand{\newcommenter}[3]{%
  \newcommand{#1}[1]{%
    \textcolor{#2}{\small\textsf{[{#3}: {##1}]}}%
  }%
}
\definecolor{darkgreen}{rgb}{0,0.7,0}
\newcommenter{\akh}{purple}{AKH}

% AMSTHM Setup
\newtheorem{thm}{Theorem}
\newtheorem{lem}{Lemma}
\newtheorem{cor}{Corollary}
\newtheorem{conj}{Conjecture}
\newtheorem{inv}{Invariant}
\theoremstyle{definition}
\newtheorem{defn}{Definition}

\author{Andrew K. Hirsch}
\title{Outreach and Diversity Statement}
\date{}

\bibliographystyle{plainnat}

\begin{document}
\maketitle

I was lucky enough to grow up in a very diverse neighborhood: the two largest ethnic groups were Hispanic and Vietnamese, but there were a good number of white and black residents as well.
This gave me an appreciation for the cultural wealth created by diversity.
I want to do what I can to bring that to the places where I work and live for the rest of my life as well.
Moreover, growing up gay in Texas in the 90s and 2000s, I struggled with my sexuality for a good portion of my life.
While I am now happily out and visible, that experience gave me a lot of empathy with people trying to figure out their identity.

These experiences spurred me to work on outreach, especially outreach which increases diversity, my entire career.
My focus has been in three areas: outreach in elementary schools, science communication, and conference organization.
In every case, my eye has been on how my work affects diversity.

Scientists often neglect elementary-school outreach, despite the incredible impact it can have.
Many students never experience computer science at all until high school or even college.
When they do, it is like their experience with math: boring, stodgy, and rote.
This especially harms girls and minority students, who are pressured by those around them not to like ``nerdy'' subjects.
By giving students fun and creative experiences with computing and math, we can inoculate them from believing that our subjects are only accessible to white men.
I started working in elementary schools my second year of my Ph.D. through Cornell's Center for Community Engagement.
I began spending two afternoons a week teaching the principles of programming to students in grades 2-5 after school.
After a year, that program was sadly not renewed.
However, I continued working in elementary schools through Cornell's math department.
Every year, a group of us would go to classrooms for grades 2-6 and teach a math class designed to be fun and creative.
Our lesson plans included crafts and discovery, and were often looked forward to by both teachers and students for the entire year.
I plan to bring elementary-school outreach to my next job, and continue providing students with a fun and refreshing experience with computing and math.

Elementary outreach helps by inoculating students against the stereotypes of computing and math as stodgy and rote, but public attitude continues to bombard students with these stereotypes.
This means that students come into computer-science classrooms with misinformed ideas about what computer scientists and programmers do which can repel women and minority students.
Moreover, public misunderstanding of computing can lead to bad public policy which affects all of us, and affects minorities disproportionately.
To combat this, I volunteered this year as a science communicator with Electronic Frontiers Georgia and the Electronic Frontiers Foundation at the Electronic Frontiers Forums held yearly as part of Dragoncon, a large convention held every year in Atlanta, Georgia.
As part of this volunteering role, I explained both basic computer-security concepts and the software-engineering issues in the Oracle v. Google case to a general audience.
I am already preparing for my role next year volunteering not only with EFF, but also with the science-communication team at Dragoncon.

While working with elementary schools and science communicators can help change how computer science is perceived by students in our classrooms, it does not directly impact the research community.
However, by disrupting the ``normal'' conference process, COVID-19 has put our community in a crossroads regarding how we communicate our research findings.
While the programming-languages community has long been on the forefront of conference design, we have recently been forced, along with every other research community, into a radical experiment with virtual conferences.
This has been a mixed blessing: many more people can attend virtual conferences from much more diverse backgrounds; however, engagement at the conferences they can attend is much lower.

As we go back to in-person conferences, we want to keep the diversity while also increasing engagement.
In many different forums, members of our community have defended many hypotheses about how best to do so, ranging from ``stay completely virtual'' to ``go back to 100\% in-person'' with every form of hybrid in-between.
I have also floated hypotheses, joining a group of students at every pre-faculty stage in a researchers career to write a post on the SIGPLAN blog discussing options.
However, I was alarmed to find how little data was being collected allowing these hypotheses to be tested.
After bringing this situation to the attention of the SIGPLAN executive committee, I have been asked to organize and chair an ad-hoc committee exploring how virtualization decisions impact diversity and engagement.
That committee is now beginning the work of organizing the data that we have and collecting more information as we go into the future.

\end{document}

%%% Local Variables:
%%% mode: latex
%%% TeX-master: t
%%% eval: (setenv "TEXINPUTS" ".::$TEXMF/tex/::./latex-pl-syntax/")
%%% End:
