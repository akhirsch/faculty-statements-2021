\documentclass{article}

\usepackage[utf8]{inputenc}
\usepackage[T1]{fontenc}
\usepackage[a4paper, top=0cm, margin=2cm, bottom=2cm]{geometry}
\usepackage[numbers,sort&compress,square]{natbib}
\usepackage{amsmath}
\usepackage{amssymb}
\usepackage{amsthm}
\usepackage{bbm}
\usepackage{bbold}
\usepackage{stmaryrd}
\usepackage{mathtools}
\usepackage{mathpartir}
\usepackage[dvipsnames]{xcolor}
%\usepackage{pl-syntax}
\usepackage{xspace}
\usepackage{suffix}
\usepackage{turnstile}
\usepackage{multicol}

\usepackage[hidelinks]{hyperref}


% Notes
\newcommand{\uncertain}[1]{{\color{red} #1}\xspace}
\newcommand{\newcommenter}[3]{%
  \newcommand{#1}[1]{%
    \textcolor{#2}{\small\textsf{[{#3}: {##1}]}}%
  }%
}
\definecolor{darkgreen}{rgb}{0,0.7,0}
\newcommenter{\akh}{purple}{AKH}

% AMSTHM Setup
\newtheorem{thm}{Theorem}
\newtheorem{lem}{Lemma}
\newtheorem{cor}{Corollary}
\newtheorem{conj}{Conjecture}
\newtheorem{inv}{Invariant}
\theoremstyle{definition}
\newtheorem{defn}{Definition}

\author{Andrew K. Hirsch}
\title{Outreach and Diversity Statement}
\date{}

\bibliographystyle{plainnat}

\frenchspacing
\usepackage{newtxtext}
\usepackage[T1]{fontenc}

\begin{document}
\maketitle

I was lucky enough to grow up in a very diverse neighborhood: the two largest ethnic groups were Hispanic and Vietnamese, but there were a good number of black and white residents as well.
This gave me an appreciation for the cultural wealth created by diversity.
I want to bring that wealth to the places where I work and live for the rest of my life.
Moreover, growing up gay in Texas in the 90s and 2000s, I struggled with my sexuality for a good portion of my life.
While I am now happily out and visible, that experience gave me a lot of empathy with people trying to figure out their identity.

These experiences have spurred me to work on outreach, especially outreach which increases diversity, my entire career.
My focus has been in three areas: outreach in elementary schools, science communication, and conference organization.
In every case, my eye has been on how my work affects diversity.

Scientists often neglect elementary-school outreach, despite the incredible impact it can have.
Many students never experience computer science at all until high school or even college.
When they do, it is all to often like their experience with math: boring, stodgy, and rote.
This especially harms girls and minority students, who are pressured by those around them not to like ``nerdy'' subjects.
By giving students fun and creative experiences with computing and math, we can inoculate them from believing that our subjects are only accessible to white men.
I started working in elementary schools through Cornell's Center for Community Engagement during the second year of my Ph.D.
I spent two afternoons a week teaching the principles of programming to students in grades 2-5 after school.
After a year, Cornell decided to direct its resources elsewhere, sadly meaning that the program was not renewed.
However, I continued working in elementary schools through the math department.
Every year, I would join other volunteers in classrooms for grades 2-6 and teach a math class which included discovery, fun, and even crafts.
Both students and teachers often looked forward to our lessons for the entire year.

Elementary outreach inoculates students against the stereotypes of computing and math as stodgy and rote, but public attitude continues to bombard students with these stereotypes.
This means that students come into computer-science classrooms misinformed about what computer scientists and programmers do which can repel women and minority students.
Moreover, public misunderstanding of computing can lead to bad public policy, which affects minorities disproportionately.
To combat this, I volunteered this year as a science communicator with Electronic Frontiers Georgia and the Electronic Frontiers Foundation at the annual Electronic Frontiers Forums held as part of DragonCon, a large convention in Atlanta, Georgia.
As part of this volunteering role, I explained both basic computer-security concepts and the software-engineering issues in the Oracle v. Google case to a general audience.
I am already preparing for my role next year volunteering not only with EFF, but also with the broader science-communication team at DragonCon.

While working with elementary schools and science communicators can help change how computer science is perceived by students in our classrooms, it does not directly impact the research community.
However, COVID-19 has disrupted the ``normal'' conference process, presenting a new opportunity to decide how we communicate our research.
We want to keep the increased diversity of virtual conferences while also regaining engagement of in-person conferences.
In many different forums, members of our community have defended many hypotheses about how best to design our conferences with these goals in mind.
Suggestions range from ``stay completely virtual'' to ``go back to 100\% in-person,'' with every form of hybrid in between.
I have been an active participant in this conversation, writing a post on the SIGPLAN blog with a group of researchers represening every pre-faculty stage of a research career.
In this process, I was alarmed to find how little data was being collected to test these hypotheses.
After bringing this situation to the attention of the SIGPLAN executive committee, I have been asked to organize and chair an ad-hoc committee exploring how virtualization decisions impact diversity and engagement.
That committee is now beginning the work of organizing what data that we already have and collecting more.

\end{document}

%%% Local Variables:
%%% mode: latex
%%% TeX-master: t
%%% eval: (setenv "TEXINPUTS" ".::$TEXMF/tex/::./latex-pl-syntax/")
%%% End:
