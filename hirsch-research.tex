\documentclass{article}

\usepackage[utf8]{inputenc}
\usepackage[T1]{fontenc}
\usepackage[letterpaper, margin=1in, bottom=1in]{geometry}
\usepackage[numbers,sort,square]{natbib}
\usepackage{amsmath}
\usepackage{amssymb}
\usepackage{amsthm}
\usepackage{bbm}
\usepackage{bbold}
\usepackage{stmaryrd}
\usepackage{mathtools}
\usepackage{mathpartir}
\usepackage[dvipsnames]{xcolor}
\usepackage{latex-pl-syntax/pl-syntax}
\usepackage{xspace}
\usepackage{suffix}
\usepackage{turnstile}
\usepackage{multicol}

\usepackage[hidelinks]{hyperref}


% Notes
\newcommand{\uncertain}[1]{{\color{red} #1}\xspace}
\newcommand{\newcommenter}[3]{%
  \newcommand{#1}[1]{%
    \textcolor{#2}{\small\textsf{[{#3}: {##1}]}}%
  }%
}
\definecolor{darkgreen}{rgb}{0,0.7,0}
\newcommenter{\akh}{purple}{AKH}

% AMSTHM Setup
\newtheorem{thm}{Theorem}
\newtheorem{lem}{Lemma}
\newtheorem{cor}{Corollary}
\newtheorem{conj}{Conjecture}
\newtheorem{inv}{Invariant}
\theoremstyle{definition}
\newtheorem{defn}{Definition}

\author{Andrew K. Hirsch}
\title{Research Statement}
\date{}

\bibliographystyle{plainnat}

\begin{document}

\maketitle

\textbf{My work focuses on the semantics and design of programming languages and logics for writing and verifying decentralized software.}
Thus, my work allows programmers to write and reason about programs with components that may be mutually-distrusting, represent different threads of computation, represent different nodes in a distributed system, or more.
To do this, I treat trust and communication as explicit parts of the design process, rather than afterthoughts.
I then keep to a key design principle: communication can only happen along trust lines.
This allows us to prove that programs and proofs are \emph{correct by construction}: communication never leaks secrets, nor does it derail computation.

My particular work in this space breaks down into three lines.
First, I build logics for modeling security policies where \textbf{trust is a first-class citizen}.
Treating trust as something which can be manipulated by the logic itself allows us to model more-realistic security policies, but makes the metatheory much more difficult to develop.
Second, I develop \textbf{new semantic foundations for language-based security}, which allows us to build a toolbox for \emph{designers} of such languages.
Language designers often have to prove theorems---such as \emph{noninterference}, the main security theorem for information-security policies---over and over again.
By taking advantage of semantic techniques, we can prove such theorems once, and then apply it to a new language by noting what language features it has, and which it does not have.
Finally, I work on \textbf{functional-concurrent deadlock-free-by-construction concurrent programming}.
This allows a programmer to program in a familiar functional style and extract out several programs---which can be used, for example, as a distributed system---which provably communicate without going wrong.

Not only is my work unified by motivations involving decentralization, communication, and trust, it also shares a foundational technical core: \emph{modal logic}.
In my work, I represent decentralized software components using \emph{modalities}.
This allows me to bring decades of work on the semantics and proof theory of modal logic to bear on the problems of decentralized software.
Moreover, modalities allow for an easy representation of communication: communication moves information from one modality (i.e., component) to another.

In another line of work, I also use modalities to represent \textbf{computational effects}, which are ways that programs interact with their environment other than receiving inputs and producing outputs.
In particular, I have used modalities to show that \textbf{strict and lazy evaluation strategies arise from effect interactions.}

\section*{Trust as a First-Class Citizen}
\citet{Abadi06} introduced the idea of reasoning about access-control policies using a logic with modalities.
He suggested that we use $p \mathrel{\textrm{says}} \varphi$ to represent the fact that the component (known in the security literature as a \emph{principal}) $p$ believes the formula $\varphi$ to be true.
He then represented trust by saying that if $p$ \emph{speaks for} $q$, then if $p$ believes $\varphi$, we can assume that $q$ also believes $\varphi$.
Abadi (and most of his successors) did not allow trust to appear as statements in the logic; instead, trust was imposed on the outside.
However, practical implementations of his idea~\citep{SchneiderWS11,SirerDBRSWWS11} required that trust be treated like any other logical formula.

My work in authorization logic gives theoretical grounding to this practical extension.
In \textit{Belief Semantics of Authorization Logic}~\citet{HirschC13}, we introduced the First-Order Constructive Authorization Logic (FOCAL), a logic for reasoning about access control with first-class trust.
We focused on building \emph{models} for this logic, which are mathematical structures we can use to give semantics to logics.
While our focus was on building models that matched practical uses such as that by \citet{SirerDBRSWWS11}, we found that standard models of such logic failed to account for intention in trust.
Thus, any principal~$p$ who happens to believe more things than a principal~$q$ trusts $q$, whether or not $p$ even knows that $q$ exists.
In that paper, we ruled out such ``accidental trust'' by fiat, but left open for future work a more-satisfactory treatment of trust in model theory.

In \textit{First-Order Logic for Flow-Limited Authorization}~\citep{HirschACAT20}, we introduced the Flow-Limited Authorization First-Order Logic.
This logic combined information-security policies with access-control policies; like FOCAL, it had trust as a first-class citizen.
However, unlike FOCAL, we proved \emph{noninterference}, the key correctness-by-construction theorem which states that beliefs of principals that $p$ does not trust does not affect $p$'s beliefs.
First-class trust poses a fundamental challenge to this theorem: who $p$ does and does not trust can change during a proof.
To see this, consider a proof that starts with the premise that either $p$ trusts $q$ or $p$ believes $\varphi$.
If we perform case analysis on this premise, then we are allowed to assume that $p$ trusts $q$ in one branch, but not the other.
Therefore, we must over-approximate trust in the noninterference theorem, leading to a less-precise result.

Current work extends the work on FLAFOL to make the noninterference less precise and several aspects of the design more practical.
Moreover, it allows reasoning about \emph{equality}, which requires more reasoning.
After all, we now must know not only whether $p$ trusts $q$, but whether $r$ trusts $s$ when $p = r$ and $q = s$.

\paragraph{Semantic Tools for Information-Security Policies}
Information-security policies tell us how data should be treated in a program.
For instance, we may choose to label some information as \textsf{Public} and other information \textsf{Private}.
Then, \textsf{Private} information should not influence \textsf{Private} observations.
\emph{Information-flow control} tries to enforce this version of \emph{noninterference}.
A common technique for enforcing noninterference labels every type in a programming language with an information-security policy, and then using the type system to enforce noninterference.

As mentioned, FLAFOL~\citep{HirschACAT20} combines reasoning about access-control policies with information-security policies.
To do this, we refine the formula $p \mathrel{\textrm{says}} \varphi$ by adding an information-flow policy describing how $p$ wants their belief treated.
We also make the notion of trust more specific: in order for $p$ to trust $q$, not only must $p$ believe what $q$ has to say, but $q$ must be willing to tell $p$.
With this simple change, we were able to show that our one noninterference theorem enforced the information-security policies.

In a program, using private information to compute a public output leaks that private information.
Private information can also leak \emph{indirectly} via an effect.
For instance, imagine a program which, if a private variable is zero, prints a string to the screen and otherwise does nothing.
Anybody who can see the screen then knows whether the variable is zero.
In order to counter this in a type system, standard practice adds a \emph{program-counter label} to the type system.
This ensures that any effect which an adversary might see occurs without regards to private data.

In \textit{Giving Semantics to Program-Counter Labels via Secure Effects}~\citep{HirschC21}, we look at the semantics of program-counter labels.
In particular, we show that program-counter labels can be given semantics via type-and-effect systems and \emph{monads}, a common structure in the semantics of effects.
Monads give semantics to effectful programs by finding equivalent \emph{pure} programs, with a new output type which gives access to effectful operations.
We state and prove the \emph{noninterference half-off} theorem, which says that if you can prove noninterference for the pure part of your language, monadic semantics allows you to get noninterference for the effectful language.
Moreover, we are able to show that under very general conditions, program-counter labels can be given semantics through this method.

In \textit{Progress-Sensitive Security Semantics with ITrees}~\citep{SilverHHCZ22}, we extend the Interaction Trees (ITrees) library~\citep{XiaZHHMPZ20} to reason about information-security policies.
Interaction trees are a Coq library for executable semantics of impure and recursive programs.
The key reasoning facility of interaction trees is \emph{equivalence up-to taus}, which gives an equivalence on ITrees.
Essentially equivalence up-to taus states that two programs are equivalent if they interact with the environment in exactly the same pattern, including provide equal outputs.
We added two new relations to the ITrees library, representing when two ITrees are equivalent according to an adversary.
In one relation, the adversary can tell when a program has gone into an infinite loop.
However, most languages can only enforce noninterference against and adversary who cannot make such a determination.
In order to accommodate those languages, we also include a weaker relation where the adversary considers an infinite loop equivalent to every other program.
\akh{This title will almost certainly change in the next few weeks.}

In \textit{Logical Relations for Higher-Order Where Declassification}~\citep{MenzHLG22}, we build models of languages with \emph{declassification}.
Declassification allows \textsf{Private} data to sometimes influence \textsf{Public} data, but only in controlled circumstances.
Thus, we enforce not noninterference, but more-complicated security guarantees.
Mixing declassification with higher-order programming makes the guarantees even harder to provide.
By using \emph{logical relations}---a common technique in the programming-languages literature---we can build better guarantees and proofs that our language enforces those guarantees.
In current work, we are extending those results to cover new methods of declassification and new security guarantees.

\paragraph{Deadlock-Free-By-Construction Higher-Order Programming}
Choreographic programming~\citep{Montesi13} is a new programming paradigm for programs with message-passing concurrency.
It allows programmers to write the entire program from one point-of-view rather than writing a single program for each thread of computation.
Then, \emph{endpoint projection} allows a compiler to produce a program for every thread of computation.
\emph{Deadlock-freedom-by-construction} guarantees that such programs never go wrong---their communication does not deadlock.
However, choreographic programming has been explored only in simple imperative settings.

In \textit{Pirouette: Higher-Order Typed Functional Choreographies}~\citep{HirschG22}, we introduce Pirouette, which extends choreographic programming to the functional setting.
Moreover, we designed our language to allow messages from any single-threaded functional base language, and provide guarantees relative to the guarantees of that language.
To support this, Pirouette has types of the form $\ell.\tau$, where $\ell$ is a \emph{location}, or a thread of computation.
We allow information to be sent from one location to another using an Alice-and-Bob--style syntax.
We then extend this with recursive functions, along with other functional programming features.
Endpoint projection compiles this to one functional program for every location, using a functional language which contains send and receive functions.
The correctness of endpoint projection, along with other properties of our language, is formalized in Coq.

Current work extends this by allowing a user to write programs which operate over any location.
This allows more code reuse, since library functions can be written once and then run on every location.
In one piece of current work, we have a pared-down functional choreographic language which we extend with polymorphism, including polymorphism over locations.
In another, we extend Pirouette to include not only polymorphism, but the ability to interact with arbitrary \emph{clients} from outside the choreographic system.


\paragraph{Strictness and Laziness from Effects}


\iffalse
\textbf{My work focuses on the semantics and design of programming languages and logics that explicitly reason about trust and communication.}
These logics allow us to reason about security policies, while the programming languages I design allow programmers to build correct-by-construction programs.
By ensuring that programs are correct---and thus, secure---we save programmers from partaking in an error-prone manual debugging process.

In order to achieve correctness-by-construction for communication and trust, we must explicitly represent the agents which perform the communication and trust.
These agents may be nodes in a distributed system; that is, computers connected by a network which communicate with each other in order to perform a common task.
Alternatively, they may be users who have provided data to the system, and who ``communicate'' whenever one's data is revealed to another.
These sorts of possibly-mutually-distrusting agents are called \emph{principals} in security.

We represent principals by using an idea from formal logic: \emph{modalities}.
In logic, a modality allows us to express that a proposition is believed by some principal.
In programming languages, a modality allows us to express that some data is owned by an principal.
To represent communication, we change modalities.
Thus, to reason about a security policy where Alice has a belief, she may communicate to Bob by transmitting that belief to him, allowing us to conclude that Bob believes the same thing.
Trust, then, limits communication.
By ensuring that communication only happens along trust lines, we enforce a basic security policy: \emph{noninterference}.

Logics with modalities, trust, and communication are known as \emph{authorization logics}.
Authorization logics allow us to reason about the consequences of security policies.
My work in logic focuses on the challenge of making trust a first-class citizen.

If we identify a principal with how their data should be treated, we can build languages with \emph{information-flow control}.
Such languages allow us to annotate data with \emph{confidentiality} and \emph{integrity} policies, expressing who may learn the data and who may determine the data, respectively.
My work on information-flow control focuses on building general metatheoretic tools which designers of such languages may apply to new languages.

If we represent a principal more directly, however, we can build languages for message-passing concurrency and distributed computing.
In particular, I introduced \emph{functional choreographic programming}, which allows us to tag data with a node in a distributed system which stores the data.
Communication is then message-passing communication.
This allows us to write deadlock-free-by-construction programs.

Finally, modalities are useful even to represent things other than principals.
In particular, they have found much use in reasoning about \emph{effectful code}, or code which performs actions other than examining its inputs and building its outputs.
Not only does much of my work on information-flow control use this method to reason about effects, but I have shown how strict and lazy evaluation orders arise from combinations of modalities.
\fi

\bibliography{bibliography/main}
\end{document}

%%% Local Variables:
%%% mode: latex
%%% TeX-master: t
%%% End:
